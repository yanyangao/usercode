For the cross section measurement of $pp\to WW\to 2l 2\nu_l$
~\cite{wwanalysis}, one of the dominant backgrounds is the top-background. 
The only difference between the top-background and the $WW$ signal is the 
presence of one or two extra $b$-jets in the final states. Thus, one 
efficienct way to suppress the top-background is to apply 
jet-veto, vetoing the events with leading jet \pt\, above a thresold. 
%In the $WW$ analysis, we choose 25 GeV as the working point. 

%The jet-veto efficiency of the $WW$ signal needs to be carefully evaluated 
%for the $WW$ cross section measurement. 
The uncertainties of jet-veto signal efficiency come from 
both the matrix element calculations % uncertainties and 
and the jet reconstruction. %uncertainties. 
In the leading order matrix element calculation, 
the $pp\to WW\to 2l2\nu$ process does not contain jet.
Extra jets are primarily due to the initial 
state radition. It makes the jet energy spectrum 
sensitive to the corrections beyond the LO. 
On the jet reconstruction side, the effects from the jet energy correction 
and its uncertainties can propagate towards the jet-veto efficiency 
estimation as well. Validation of the jet response between the 
data and MC is thus required. 

In this note, we describe a partially data-driven approach to 
measure the $WW$ jet-veto signal effcicieny.
In this method, we exploit the similarity of the 
$pp\to WW\to 2l 2\nu_l$ and $pp\to Z\to 2l$ processes.
We estimate the jet-veto signal efficiency on $WW$ 
as the jet-veto efficiency of
$pp\to Z\to 2l$ on data multiplied by the $WW/Z$ jet-veto efficiency 
ratio estimated on the Monte Carlo, shown in Equation~\ref{eq:wweff}. 
Table~\ref{tab:datasets} summarizes the data and MC samples used in this note. 

%%%%%%%%%%%%%%%%%%%%%%%
\begin{eqnarray}
\epsilon_{WW}^{data} 
%\equiv {\epsilon_Z^{data}} \times {\frac{\epsilon_{WW}^{data}}{\epsilon_Z^{data}}} 
\approx {\epsilon_Z^{data}} \times ({\epsilon_{WW}^{MC}}/{\epsilon_Z^{MC}}) = {\epsilon_Z^{data}} \times R_{WW/Z}^{MC}  
\label{eq:wweff}
\end{eqnarray}
%%%%%%%%%%%%%%%%%%%%%%%


%%%%%%%%%%%%%%%%%%%%%5
\begin{table}[htbp]
\caption{List of datasets} 
\begin{center}
\label{tab:datasets}
\begin{tabular}{c|c}
\hline
\hline
\multicolumn{2}{c}{Collision Data} \\ 
Luminosity & Dataset \\ %& nEvents & note \\
\hline
\vspace{-3mm}   &    \cr
\multirow{5}{*} {3.1} & /EG\_Run2010A-PromptReco-v4\_RECO/ \\
& /EG\_Run2010A-PromptReco-v4\_RECO/ \\
& /EG\_Run2010A-Jul16thReReco-v2\_RECO/ \\
& /EG\_Run2010A-Jun14thReReco\_v1\_RECO/ \\
& /MinimumBias\_Commissioning10-SD\_EG-Jun14thSkim\_v1\_RECO/ \\
\hline
\multicolumn{2}{c}{Monte Carlo} \\
\hline
\multirow{3}{*} {Pythia} & /Zee\_Spring10-START3X\_V26\_S09-v1/\\ 
& /Zmumu\_Spring10-START3X\_V26\_S09-v1/\\ 
& /WW\_Spring10-START3X\_V26\_S09-v1 \\ \hline
\multirow{2}{*} {Madgraph} & /ZJets-madgraph\_Spring10-START3X\_V26\_S09-v1 \\
& /VVJets-madgraph\_Spring10-START3X\_V26\_S09-v1/ \\ \hline
\multirow{2}{*} {MC@NLO} & /Zgamma\_ee\_M20-mcatnlo\_Spring10-START3X\_V26\_S09-v1/ \\
& /Zgamma\_mumu\_M20-mcatnlo\_Spring10-START3X\_V26\_S09-v1/ \\
& /WWtoEE-mcatnlo\_Spring10-START3X\_V26\_S09-v1/ \\
& /WWtoEPlusMuMinus-mcatnlo\_Spring10-START3X\_V26\_S09-v1/ \\
& /WWtoEPlusTauMinus-mcatnlo\_Spring10-START3X\_V26\_S09-v1/ \\
& /WWtoMuMu-mcatnlo\_Spring10-START3X\_V26\_S09-v1/ \\
& /WWtoMuPlusEMinus-mcatnlo\_Spring10-START3X\_V26\_S09-v1/ \\
& /WWtoMuPlusTauMinus-mcatnlo\_Spring10-START3X\_V26\_S09-v1/ \\
& /WWtoTauTau-mcatnlo\_Spring10-START3X\_V26\_S09-v1/ \\
& /WWtoTauPlusEMinus-mcatnlo\_Spring10-START3X\_V26\_S09-v1/ \\
& /WWtoTauPlusMuMinus-mcatnlo\_Spring10-START3X\_V26\_S09-v1/ \\
\hline
\hline
\end{tabular}
\end{center}
\end{table}
%%%%%%%%%%%%%%%%%%%%%5
