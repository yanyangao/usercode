In this section, we describe the $WW/Z$ jet-veto efficiency 
ratio $R_{WW/Z}$ (Equation~\ref{eq:wweff}) estimation. 
The $WW$ production has a different kinematic from the $Z$, as 
the energy scale in the former is much larger than the latter 
(see Figure~\ref{fig:wwz_pthat}, left).
%The virtual $Z$ propagator in the $WW$ production 
%has larger energy transfer than the on-shell $Z$ production, 
This indicates that the incoming partons $q\bar{q}$ in the $WW$ 
production have a larger energy on average than the $Z$ productions. 
This results in a harder ISR parton energy spectrum, 
shown in Figure~\ref{fig:wwz_pthat} (right). 


Figure~\ref{fig:wwzratio}-\ref{fig:wwzratio_JEC} show 
the $WW/Z$ jet-veto efficiency ratio dependence on the 
leading jet \pt\,, using the uncorrected and corrected jets respectively. 
The predictions from the Madgraph agree with the MC@NLO within 1-2\%.
%at a difference level $\approx 1$\%.% at the jet \pt cut above 20 GeV. 
The predictions from the Pythia MC differ from the 
other two in the order of 10\% on average. 
Note that the jet-veto efficiencies in the $Z$ control region 
from Pythia and Madgraph MC samples agree within 3\% (section~\ref{sec:zeff}). 
The large Pythia/Madgraph difference in the $R_{WW/Z}$ 
is due to the different leading jet \pt\, spectrum in the 
$WW$ productions. 
%is due to the different handling of ISR partons in $WW$ production. 
More specifically, Pythia predicts a much softer 
jet energy spectrum than Madgraph in the $WW$ production. 

In Pythia, the matrix element calculation for $WW$ production 
includes only the leading order contributions. 
The ISR is modeled through the parton 
showering in the soft-collinear limit. 
On the other hand, both the Madgraph and mc@nlo MCs 
take into account upto 1 parton ISR in the matrix element calculation.
The large Pythia/Madgraph difference could be due to the 
imperfect modelling of the ISR in Pythia parton showering. 
%the LO in Pythia are not properly modelled 
%through the parton showering for the process $pp\to WW$. 
%The good agreement between those two samples is not surprising. 
The good agreement between the Pythia and Madgraph in the 
$Z$ jet-veto efficiency is likely because the Pythia parton 
showering is tuned well on $Z$ region using data. 


%To disentangle the jet reconstruction differences, we studied the 
%$WW/Z$ jet-veto efficiency ratio dependence on both the generator level 
%jet energy and the parton energy, shown in Figure~\ref{fig:wwzratio_mcfm}. 
%The large difference of the Pythia predictions from Madgraph and mc@nlo 
%persists at the generator level. 
%We use the MCFM\cite{ref:mcfm} MC generator to probe the $WW/Z$ jet-veto 
%efficency ration dependence on the matrix element level parton energy. 
%The events are generated with up to 1 extra parton in the final states. 
%The parton $\pt$ is taken as the magnitude of the recoiling four or two 
%leptons \pt in the $WW\to 2l2\nu$ or $pp\to Z\to 2l$ decays. 


%To see the theorectical uncertainties of the NLO calcuations, 
%we change the normalization ($\mu_R$) and factorization ($\mu_F$) 
%scales in the calcuation to $1/2$ and $2$ times the conventional 
%chosen values. 
%The larger difference from the values obtained with the nominal 
%$\mu_R$ and $\mu_F$ can be quoted as the theorectial uncertainties. 

Therefore, we choose the Madgraph as the MC sample to calculate the 
%nominal value of 
$WW/Z$ jet-veto efficiency ratio $R_{WW/Z}^{MC}$. 
We assign half the Pythia/Madgraph difference  as its systematic error. 
The $R_{WW/Z}^{MC}$ results are tabulated 
Table~\ref{tab:wwzratio_results}-\ref{tab:wwzratio_jec_results} 
using the uncorrected and corrected PFJets 
for three different jet energy cut (20, 25, 30) GeV. 


%%%%%%%%%%%%%%%%%%%%%%%
\begin{figure}[htb]
\begin{center}
\begin{tabular}{c}
\epsfig{figure=figures/pthat.eps, width=3in} 
\epsfig{figure=figures/wwz_genjet_pt.eps, width=3in} 

\end{tabular}
\caption{
Left: The $\hat{s}$ for the $pp\to WW$ (Blue) and $pp\to Z$ (Red). 
Right: The GenJet max jet energy spectrum for the $pp\to WW$ (Blue) and $pp\to Z$ (Red). 
}
\label{fig:wwz_pthat}
\end{center}
\end{figure}
%%%%%%%%%%%%%%%%%%%%%%%



%%%%%%%%%%%%%%%%%%%%%%%
\begin{figure}[htb]
\begin{center}
\begin{tabular}{ccc}
\epsfig{figure=figures/wwzratio_hmaxPFJetPt_all.eps, width=3in}
\epsfig{figure=figures/wwzratio_hmaxJPTPt_all.eps, width=3in}
\end{tabular}
\caption{
Left: The $WW/Z$ jet-veto efficiency ratio $R_{WW/Z}$ using the PFJet;
Right: The $WW/Z$ jet-veto efficiency ratio $R_{WW/Z}$ using the PFJet;
}
\label{fig:wwzratio}
\end{center}
%\end{figure}
%%%%%%%%%%%%%%%%%%%%%%%
%%%%%%%%%%%%%%%%%%%%%%%
%\begin{figure}[htb]
\begin{center}
\begin{tabular}{ccc}
\epsfig{figure=figures/wwzratio_hmaxPFJetPt_all_JEC.eps, width=3in}
\epsfig{figure=figures/wwzratio_hmaxJPTPt_all_JEC.eps, width=3in}
\end{tabular}
\caption{
Left: The $WW/Z$ jet-veto efficiency ratio $R_{WW/Z}$ using the corrected PFJet;
Right: The $WW/Z$ jet-veto efficiency ratio $R_{WW/Z}$ using the corrected PFJet;
}
\label{fig:wwzratio_JEC}
\end{center}
\end{figure}
%%%%%%%%%%%%%%%%%%%%%%%

%%%%%%%%%%%%%%%%%%%%%%%
%\begin{figure}[htb]
%\begin{center}
%\begin{tabular}{c}
%\epsfig{figure=figures/wwzratio_hmaxGenJetPt_all.eps, width=3in}
%\epsfig{figure=figures/wwzratio_Parton.eps, width=3in}
%\end{tabular}
%\caption{
%Left: The $WW/Z$ jet-veto efficiency ratio $R_{WW/Z}$ using the GenJet;
%Right: The $WW/Z$ jet-veto efficiency ratio $R_{WW/Z}$ using the 
%parton \pt. The parton four-momentum is calculated as the 
%sum of the four (two) leptons in the case of $WW\to 2l2\nu_l$ ($Z\to 2l$).
%}
%\label{fig:wwzratio_mcfm}
%\end{center}
%\end{figure}
%%%%%%%%%%%%%%%%%%%%%%%


%%%%%%%%%%%%%%%%%%%%%5
\begin{table}[htbp]
\caption{ The $WW/Z$ jet-veto signal efficiency ratio using the uncorrected PFJet. 
The efficiency ratio is quoted as the value using PFcJets based on Madgraph MC sample. 
The uncertainty on $R_{WW/Z}^{M}$ is taken as half of the difference 
between Pythia and Madgraph.}
\begin{center}
\label{tab:wwzratio_results}
\begin{tabular}{cc}
\hline
\hline
Jet-veto Cut (GeV) & $\sigma R_{WW/Z}^{MC}$ (\%) \\
    20 &  76.3$\pm$ 0.7$\pm$ 6.2 \\
    25 &  79.1$\pm$ 0.6$\pm$ 4.9 \\
    30 &  80.9$\pm$ 0.6$\pm$ 4.2 \\
\hline
\hline
\end{tabular}
\end{center}
%\end{table}
%%%%%%%%%%%%%%%%%%%%%5
%%%%%%%%%%%%%%%%%%%%%5
%\begin{table}[htbp]
\caption{ The $WW/Z$ jet-veto signal efficiency ratio using L2L3 corrected PFJet. 
The efficiency ratio is quoted as the value using PFcJets based on Madgraph MC sample. 
The uncertainty on $R_{WW/Z}^{M}$ is taken as half of the difference 
between Pythia and Madgraph.}
\begin{center}
\label{tab:wwzratio_jec_results}
\begin{tabular}{cc}
\hline
\hline
Jet-veto Cut (GeV) & $\sigma R_{WW/Z}^{MC}$ (\%) \\
    20 &  74.9$\pm$ 0.7$\pm$ 6.5 \\
    25 &  77.5$\pm$ 0.7$\pm$ 5.7 \\
    30 &  79.7$\pm$ 0.6$\pm$ 4.8 \\
\hline
\hline
\end{tabular}
\end{center}
\end{table}
%%%%%%%%%%%%%%%%%%%%%5

