\subsection{Two-vertex Method Description}

To measure the primary vertex resolution, we measure the primary vertex 
position of each event with two independent sets of tracks. 
We obtain the algorithm resolution by comparing the 
relative position of the two vertices in $x$, $y$ and $z$. 
The distribution of the difference in the fitted vertex positions 
can then be used to extract the resolution and pulls. 
The method proceeds as follows. 

%%%%%%%%%%%%%%%%%%%%
\begin{enumerate}
\item Divide tracks used in the primary vertex in two independent sets (A and B)
%%%%%%%%%%%%%%%%
\begin{itemize}
\item
We take tracks used the ``offlinePrimaryVertices'' and sort them in descending order of \pt. 
\item 
Starting from highest-\pt track, we take pairs of tracks and 
randomly choose which track goes into which set (A or B). This 
ensures that there is no asymmetry of the track \pt\, spectrum in 
the two vertices. If the number of the tracks are odd, we drop 
the lowest-\pt track. 
\end{itemize}
%%%%%%%%%%%%%%%%
We monitor the basic track distributions (\pt, $\eta$, $\phi$, $dxy$ and $dz$) 
of the split track sets. Figure~\ref{fig:splittrack} show the comparison 
of these variables between the original tracks and split track sets A and B in 
the mininum bias collision data taken at $\sqrt{s}=$900GeV. 
We see that the split track sets have consistent kinematics as the 
original tracks. 

%%%%%%
\item Vertex fit each trackset (A and B) independently using ``offlinePrimaryVertices'' algorithm.

\item Compare the primary vertex position between the two fit vertices and extract resolution and pull. 

When comparing the two fitted vertices, it is important that 
the number of tracks used in the final fitted vertices and the 
$\Sigma\pt$ and $z$ of the two vertices do not differ by too much. 
As we sort the tracks according the \pt\, prior to the splitting 
step, the difference of the $\Sigma\pt$ between the two vertices 
are negligible. We select on the following two variables to ensure the 
number of tracks and vertex $z$ difference of the two vertex is small:
\begin{itemize}
\item Relative difference of the number of tracks used in the two vertices, 
$|\frac{\tt{nTrk(vtx1)-nTrk(vtx2)}}{\tt{nTrk(vtx1)+nTrk(vtx2)}}|<0.1$
\item The separation signficance in z, defined as 
$\frac{|z(vtx1)-z(vtx2)|}{\tt{max}(\sigma z(vtx1),\sigma z(vtx2))}>5$
\end{itemize}
Figure~\ref{fig:anaselection} shows the distribution of these 
two variables before applying the cuts. The selection efficiency of the two 
cuts is found to be 74\% in both data and MC. 

We analyze the difference between the two selected matching vertices and 
and extract the resolution and pull dependence of the number of tracks 
used in the primary vertex as follows. 
%%%%%%%%%
\begin{itemize}
\item
The difference between the coordinates of the two vertices in $x$, $y$ 
and $z$ is histogrammed for each number of tracks. We fit a single 
gaussian to each distribution in the range of $\pm2\times$RMS. 
The resolution is defined as the gaussian width $\sigma$ divided by $\sqrt{2}$.
\item
The pull distribution at each number of track bin is filled with the 
quantity ${x_1-x_2}\over\sqrt{\sigma x_1^2+\sigma x_2^2}$, where 
$x_i$ and $\sigma x_i$ are the vertex position and errors of the two vertices.
We fit a single gaussian to each distribution in the range of $\pm2\times$RMS.
\end{itemize}
%%%%%%%%%
\end{enumerate}

Before moving forward, it is important to point out that this method rovides an estimate 
of the resolution and pulls comparing vertices formed 
with average number of tracks that are about half of the number of tracks 
that are used in the actual primary vertex fitter. 
As the intrinsic vertex resolution is well known to decrease with the 
number tracks used in the primary vertex fitter. Therefore, the average 
resolutin obtained from this method will always be larger than the actual 
resolution in a given sample. More details on the systematic uncertainties 
are described in section~\ref{sec:systematics}.

%%%%%%%%%%%%%%%%%%%%%%%
\begin{figure}[htb]
\begin{center}
\begin{tabular}{ccc}
\epsfig{figure=figures/trkPtPV.eps,height=2.0in}
\epsfig{figure=figures/trkEtaPV.eps,height=2.0in}
\epsfig{figure=figures/trkPhiPV.eps,height=2.0in}\\
\epsfig{figure=figures/trkDxyCorrPV.eps,height=2.0in}
\epsfig{figure=figures/trkDzPV.eps,height=2.0in}
\end{tabular}
\caption{\sl 
Track \pt, $\eta$, $\Phi$, $dxy$ and $dz$ distributions of the 
tracks used in the primary vertex. 
Data is shown with red dots while MC is shown as histogram filled in blue.
}
\label{fig:splittrack}
\end{center}
\end{figure}
%%%%%%%%%%%%%%%%%%%%%%%
%%%%%%%%%%%%%%%%%%%%%%%
\begin{figure}[htb]
\begin{center}
\centerline{
\epsfig{figure=figures/nTrkDiff.eps,height=2.5in}
\epsfig{figure=figures/twovtxzsign.eps,height=2.5in}
}
\caption{\sl
Left: The relative difference of number of tracks used in 
the two vertex fit; Right: The $z$-signficance of the 
two vertex fit. Please see text for the definition. 
Data is shown with red dots while MC is shown as histogram filled in blue.
}
\label{fig:anaselection}
\end{center}
\end{figure}
%%%%%%%%%%%%%%%%%%%%%%%

\subsection{Resolution and Pulls in Different \pt\, Ranges}

