In conclusion, we develop a data-driven algorithm, referred as two-vertex method, for the measurement of primary vertex resolution. We have validated the 
method on the MC simulated data. The results obtained using the 
two-vertex method are consistent with the results obtained by comparing 
the reconstructed vertex postion and the simulated position. 

Applying the two-vertex method to the mininum bias data taken at 
$\sqrt{s}=$900 GeV and 2360 GeV, reprocessed in the Dec19th cycle. 
The resolution and pull distributions versus the number 
of tracks used in the primary vertex fitter are consistent when applying 
the two-vertex method to data and MC. 

The primary vertex resolutions are measured to be much smaller than 
the beam width measurement on the mininum bias data taken at both 
$\sqrt{s}=$ 900 and 2360GeV.
More specifically, the primary vertex resolution 
are measured to be close to 50 $\mu m$ in both $x$ and $y$, 
while the transverse beam width is around 290 $\mu m$. 
Similarly, the primary vertex resolution in $z$ is measured as close to 
70 $\mu m$ with more than 10 tracks, while the longitudinal beam width 
is around $3$ cm. 



