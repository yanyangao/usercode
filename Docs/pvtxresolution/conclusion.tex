In conclusion, we have developped a data-driven algorithm, 
referred as two-vertex method, to measure the primary vertex resolution. 
We have validated the method on the MC simulated data. 
The results obtained using the two-vertex method are 
consistent with the results obtained by comparing 
the reconstructed and simulated vertex positions. 

We apply the two-vertex method to the mininum bias data taken at 
$\sqrt{s}=$900 GeV and 2360 GeV, reprocessed in the Dec19th cycle. 
The resolution and pull distributions versus the number 
of tracks used in the primary vertex and the average \pt\, of those tracks 
are consistent with the results obtained by applying the two-vertex 
method on the MC data. 

The primary vertex resolutions are measured to be much smaller than 
the beam width in both 900 and 2360GeV Data.
More specifically, the primary vertex resolution in $x$ and $y$
are measured to be close to 50 $\mu m$, while the transverse beam width is 
measured to be $\sigma_x \sim 200\mu m$ and $\sigma_y \sim 120\mu m$ 
in the 900GeV data and $\sigma_{x,y}\sim 120\mu m$ in the 
2.36 TeV data.
The primary vertex resolution in $z$ is measured as close to 
70 $\mu m$ with more than 10 tracks, while the beam length in $z$
is found to be $\sigma_z\sim 4$ cm in the 900GeV data and 
$\sigma_z\sim 2.8$ cm in the 2.36TeV data.




