This analysis is performed with the data collected at the center-of-mass energy of 900\,GeV and 2360\,GeV. 

\subsection {Data Samples}

We use {\em December $19^{\rm th}$} reprocessing of the collected data and MC datasets:
\begin{itemize}
\item /MinimumBias/BeamCommissioning09-BSCNOBEAMHALO-Dec19thSkim\_336p3\_v1/
\item /MinBias/Summer09-STARTUP3X\_V8K\_900GeV-v1/
\item /MinBias/Summer09-STARTUP3X\_V8L\_2360GeV-v1/
\end{itemize} 

For the collision dataset, the alignment parameters of the silicon tracker were computed with about 
two million of cosmic ray tracks collected in November 2009 and the nominal values of the 
alignment parameter errors (APE) have been used in the reconstruction.
The simulated events used in this note are minimum-bias events produced with PYTHIA 6.4 \cite{Pythia} 
event generator at center-of-mass energies of 900\,GeV and 2360\,GeV and processed 
with a simulation of the CMS detector response based on GEANT 4 \cite{Geant}. 
The applied misalignment, miscalibration 
and dead channel map correspond to the expected start-up conditions. 
The longitudinal distribution of the primary collision vertices has been tuned to match the real
data. The signal in the silicon strip tracker was simulated in {\em peak} mode 
in agreement with the mode used in the readout chips, during the data taking~\cite{CMS_NOTE_2009_021}.  

To select the minimum bias collsion events, the skimmed collision dataset has the following 
technical trigger bits requirements: 
%the presence of at 
%least one BSC hit on both side of the CMS detector along the beam line 
%in coincidence with signals from both the BPTX detectors which indicate 
%the passage of a bunch from both beams at the interaction point. 
%The background from non-collision
%events has been reduced by requiring the absence of any coincidence in the BSC
% compatible with beam-produced particles which cross the detector
%from one end to the other. The trigger requirement is as follows:
\begin{itemize}
\item BSC trigger: technical trigger bit 40 or 41 
\item Veto BeamHalo: Triggers: 36, 37, 38, 39
\end{itemize}

In addition, we apply technical trigger bit 0 to pick up the correct bunch crossing for data. 
Note that this is not simulated in MC. 

\subsection{Event Selection}

To reduce further the background from non-collision events and to select useful events for tracking studies, 
we have the following event selections: 
\begin{itemize}
\item At least one real primary vertex reconstructed
\item fraction of {\em highPurity} tracks larger than 20\% if the 
number of reconstructed tracks is larger than 10.
\end{itemize}

Table~\ref{tab:datasets} shows the selection effcidency and final number of events analysis 
in each datasets.

%%%%%%%%%%%%%%%%%%%%%5
\begin{table}[htbH]
\begin{center}
\caption{List of datasets and selection efficiency. 
The selection efficiency on n-th colomn is obtained in addition to the cuts in the preceding n-1 columns.
\label{tab:datasets}}
\begin{tabular}{ccccc}
\hline
\hline
 Dataset name & Trigger & primary vertex & track high purity &  Events left \\
\hline
900\,GeV Dec19thSkim& 100\% & 94\% & 100\% & 260,000 \\
900\,GeV STARTUP3X\_V8K & 67\% & 93\% & 100\% & 256,000 \\
2360\,GeV Dec19thSkim & 100 \% & 95\% & 100\% & 12,959 \\
2360\,GeV STARTUP3X\_V8L & 68\% & 94\% & 100\% & 12,491 \\
\hline
\hline
\end{tabular}
\end{center}
\end{table}
