In this section, we discuss briefly the systematic uncertainties in 
the primary vertex resolution measurement with two-vertex method.

\subsection{Effect of the Alignment Scenario}
\label{sub:alignsys}

The primary vertex resolution depends on the track impact parameter 
resolution, which is affected by the tracker alignment. 
At the early stage of data taking, the alignment senario may 
not describe the actual value. 
The systematic uncertainties on the track impact parameters induced 
by the misalignment will be translated to the systematic uncertainties
of primary vertex resolution. 

To estimate the effect of misalignment on the primary vertex resolution, 
we repeat the study on the MC data simulated with perfectly 
aligned tracker and with the alignment position error (APE) set to 0. 
Figure~\ref{fig:resvsntrk-aligncheck}-~\ref{fig:pullvsntrk-aligncheck} 
show the resolution and pulls versus number of tracks results comparing 
900 GeV collision data, MC with misalignment scenario matching 
the data taking and the MC with ideal geometry and APE.

From this study, we estimate systematic uncertainty due to the mis-alignment 
is within 5 $\mu m$ when the number of tracks exceed 5. 
The difference in resolution can be as large as around 20-30 $\mu m$ when 
number of tracks are less than 5. 
This difference also shows up in the corresponding bins of 
pull distributions. Besides the expected larger uncertainties on these 
low number of tracks vertices, we also find that the single 
gaussian fit does not describe the residual well. 
It is sensitive to the fit range. 
Nevertheless, in all number of track bins, the mis-aligned systematic 
uncertainties are within 10\% of the resolution itself. 

%%%%%%%%%%%%%%%%%%%%%%%
\begin{figure}[htb]
\begin{center}
\centerline{
\epsfig{figure=figures/ResX_IdealMC.eps,height=2.0in}
\epsfig{figure=figures/ResY_IdealMC.eps,height=2.0in}
\epsfig{figure=figures/ResZ_IdealMC.eps,height=2.0in}
}
\caption{\sl
Primary vertex resolution as a function of the number of tracks used in the 
fitted vertex, comparing the collison data taken at $\sqrt{s}=900$GeV (Black), 
mininum bias MC with startup (green) and ideal (red) alignment. }
\label{fig:resvsntrk-aligncheck}
\end{center}
%\end{figure}
%%%%%%%%%%%%%%%%%%%%%%%
%%%%%%%%%%%%%%%%%%%%%%%
%\begin{figure}[htb]
\begin{center}
\centerline{
\epsfig{figure=figures/PullX_IdealMC.eps,height=2.0in}
\epsfig{figure=figures/PullY_IdealMC.eps,height=2.0in}
\epsfig{figure=figures/PullZ_IdealMC.eps,height=2.0in}
}
\caption{\sl
Fitted pulls from primary vertex distribution 
as a function of the number of tracks used 
in the fitted vertex,  comparing the collison data taken at $\sqrt{s}=900$GeV (Black), 
mininum bias MC with startup (green) and ideal (red) alignment. }
\label{fig:pullvsntrk-aligncheck}
\end{center}
\end{figure}
%%%%%%%%%%%%%%%%%%%%%%%


\subsection{Effect of the split vertex weight difference}
\label{sub:weight}

In the Adaptiver filter, all tracks in a given cluster are used in the 
vertex fit. The outlier tracks with larger distance to the vertex postion are 
down weighted significantly thus contribute little to the fitted vertex 
positions or errors. 
Thus as we compare the two split vertices, though the number of tracks do not 
differ, the number of tracks with high weights ($>0.5$) (or the numbers of degree of freedom) may differ. 
In that case, the difference between the two vertex positions does not 
represent the vertex error in either vertex. 

To cross check this effect, we repeat the study with resolution 
and pull versus the number of high weighted tracks instead of the total number 
of tracks. 
Figure~\ref{fig:resvsntrk-HW}-~\ref{fig:pullvsntrk-HW} show the comparison 
between the two approaches. 
The difference between the two methods are expected as the number of high 
weighted tracks is always smaller than the number of total tracks in the 
vertex. On the other hand, the resolution at the stable (or tail) region 
of number of tracks are consistent with each. This indicates that 
the low weight tracks do not affect the measurement. 
Besides, the pull distributions are consistent in all number of tracks bins.

%%%%%%%%%%%%%%%%%%%%%%%
\begin{figure}[htb]
\begin{center}
\centerline{
\epsfig{figure=figures/ResX_HW.eps,height=2.0in}
\epsfig{figure=figures/ResY_HW.eps,height=2.0in}
\epsfig{figure=figures/ResZ_HW.eps,height=2.0in}
}
\caption{\sl
Primary vertex resolution as a function of the number of tracks used in the 
fitted vertex.
}
\label{fig:resvsntrk-HW}
\end{center}
%\end{figure}
%%%%%%%%%%%%%%%%%%%%%%%
%%%%%%%%%%%%%%%%%%%%%%%
%\begin{figure}[htb]
\begin{center}
\centerline{
\epsfig{figure=figures/PullX_HW.eps,height=2.0in}
\epsfig{figure=figures/PullY_HW.eps,height=2.0in}
\epsfig{figure=figures/PullZ_HW.eps,height=2.0in}
}
\caption{\sl
Fitted pulls from primary vertex distribution as a function of the number of tracks used in the 
fitted vertex.}
\label{fig:pullvsntrk-HW}
\end{center}
\end{figure}
%%%%%%%%%%%%%%%%%%%%%%%
\clearpage