In this section, we discuss briefly the systematic uncertainties in 
the primary vertex resolution measurement. 

\subsection{Effect of the Alignment Scenario}

The primary vertex resolution depends on the track impact parameter resolution,
which is affected by the tracker alignment senarios. 
At the early stage of data taking, the alignment senario may 
not describe the actual value. 
The systematic uncertainties on the tracking parameters induced 
by the misalignment will be translated to the systematic errors 
of primary vertex resolution. 

To estimate the effect of misalignment on the primary vertex resolution, 
we repeat the study on the MC dataset simulated with perfectly 
aligned tracker and with the alignment position error (APE) set to 0. 
Figure~\ref{fig:resvsntrk-aligncheck}-~\ref{fig:pullvsntrk-aligncheck} 
show the resolution and pulls versus number of tracks for the collision data, 
MC with misalignment scenario matching the data taking and the MC with 
ideal geometry and APE.

From this study, we estimate systematic uncertainty due to the mis-alignment 
is within 5 $\mu m$ where the number of tracks exceed 5. 
The difference in resolution can be as large as around 20-30 $\mu m$ 
in the case where number of tracks are less than 5. 
This difference is also reflected in the corresponding bins of 
pull distributions. On the other hand, with less than 5 tracks in the 
primary vertex, the gaussian fit does not describe the residual well. 
So the difference highly depends on the fit range. 
Nevertheless, in all number of track bins, the mis-aligned systematic 
uncertainties are within 10\% of the resolution itself. 

%%%%%%%%%%%%%%%%%%%%%%%
\begin{figure}[htb]
\begin{center}
\centerline{
\epsfig{figure=figures/ResX_IdealMC.eps,height=2.0in}
\epsfig{figure=figures/ResY_IdealMC.eps,height=2.0in}
\epsfig{figure=figures/ResZ_IdealMC.eps,height=2.0in}
}
\caption{\sl
Primary vertex resolution as a function of the number of tracks used in the 
fitted vertex.
}
\label{fig:resvsntrk-aligncheck}
\end{center}
%\end{figure}
%%%%%%%%%%%%%%%%%%%%%%%
%%%%%%%%%%%%%%%%%%%%%%%
%\begin{figure}[htb]
\begin{center}
\centerline{
\epsfig{figure=figures/PullX_IdealMC.eps,height=2.0in}
\epsfig{figure=figures/PullY_IdealMC.eps,height=2.0in}
\epsfig{figure=figures/PullZ_IdealMC.eps,height=2.0in}
}
\caption{\sl
Fitted pulls from primary vertex distribution as a function of the number of tracks used in the 
fitted vertex.}
\label{fig:pullvsntrk-aligncheck}
\end{center}
\end{figure}
%%%%%%%%%%%%%%%%%%%%%%%


\subsection{Effect of the split vertex weight difference}

In the Adaptiver filter, all tracks that are clusterized in a given cluster 
are used in the vertex fit. The outlier tracks which are away from 
the vertex position are downweighted signficantly and contribute 
little to the fit vertex positions or errors. 
Thus as we compare the two split vertices, though the number of tracks do not 
differ, the number of tracks with high weights ($>0.5$) (or the numbers of degree of freedom) may differ. 
In that case, the difference between the two vertex positions does not 
represent the vertex error in either vertex. 

To cross check this effect, we repeat the study with resolution 
and pull versus the number of high weighted tracks instead of the total number 
of tracks. 
Figure~\ref{fig:resvsntrk-HW}-~\ref{fig:pullvsntrk-HW} show the comparison 
between the two approaches. 
And we see that the resolution at the tail region between the two methods are consistent. 
And the pull distributions are consistent in all regions. 

%%%%%%%%%%%%%%%%%%%%%%%
\begin{figure}[htb]
\begin{center}
\centerline{
\epsfig{figure=figures/ResX_HW.eps,height=2.0in}
\epsfig{figure=figures/ResY_HW.eps,height=2.0in}
\epsfig{figure=figures/ResZ_HW.eps,height=2.0in}
}
\caption{\sl
Primary vertex resolution as a function of the number of tracks used in the 
fitted vertex.
}
\label{fig:resvsntrk-HW}
\end{center}
%\end{figure}
%%%%%%%%%%%%%%%%%%%%%%%
%%%%%%%%%%%%%%%%%%%%%%%
%\begin{figure}[htb]
\begin{center}
\centerline{
\epsfig{figure=figures/PullX_HW.eps,height=2.0in}
\epsfig{figure=figures/PullY_HW.eps,height=2.0in}
\epsfig{figure=figures/PullZ_HW.eps,height=2.0in}
}
\caption{\sl
Fitted pulls from primary vertex distribution as a function of the number of tracks used in the 
fitted vertex.}
\label{fig:pullvsntrk-HW}
\end{center}
\end{figure}
%%%%%%%%%%%%%%%%%%%%%%%