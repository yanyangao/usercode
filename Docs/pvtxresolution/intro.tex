
The precise estimate of the primary vertex poistion on the event basis 
is crucial for many physics measurements (lifetime, b-tagging etc) 
in p-p collision. 
It is especially important in the early data taking at 
low energy $\sqrt{s}=900$ GeV, when both beam transverse width 
and longitudinal length are large. 
It also helps in separating the interesting parton level 
\emph{hard interactions} from the huge background due to 
long distance diffractive interact ions of protons. 


Several vertex fitting algorithms~\cite{pvtxreco} are implemented 
and studied in CMS reconstruction framework. 
Among them, the \emph{Adaptive Vertex Filter} algorithm is the most robust one and hence being used in the default offline reconstruction. 
It starts from the \emph{generalTrack} track collection. 
The prompt tracks are selected with cuts on the track transverse 
impact parameter, number of hits, and the $\chi^2$ per degree of freedom. 
These selected tracks are then clustered in $z$. Clusters are
split when there is a gap over a thresold. The tracks in each
cluster are then fit with an Adaptive Vertex Fit, which is 
based on an iterative re-weighted least-squares fit based on Kalman filter. 
Each track in the vertex is assigned a weight between 0 and 1
based on their compatibility with the common vertex. 
The outlier tracks with larger distance to the vertex postion are 
down weighted significantly, which makes the algorithm robust against the 
outliers. 

The primary vertex resolution is given by the uncertainty 
reported by the primary vertex algorithm. 
It is tightly coupled to the impact parameter resolution of the input tracks. 
It also strongly depends on the number of tracks used in fitting the vertex 
and the \pt\,of those tracks. 
Since each vertex is independent and has its own uncertainty, it is not 
mathematically possible to define a primary vertex resolution for a specific sample.
Therefore, the best we can do is to try to measure an average 
resolution of an event ensemble.

In this note, we introduce a data-driven method, referred as two-vertex 
method, to measure the vertex resolution dependence on the 
number of tracks and averge \pt\, of those tracks used in the primary vertex. 
In this method, we measure the primary vertex position of each event 
with two independent sets of tracks. 
The vertex resolution is obtained by comparing the 
relative position of the two vertices in $x$, $y$ and $z$. 
The distribution of the difference in the fitted vertex positions 
can then be used to extract the resolution and pulls. 

