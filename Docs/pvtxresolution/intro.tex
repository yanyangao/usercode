The event primary vertex resolution is given by the uncertainty 
reported by the primary vertex algorithm. 
The reconstruction of the primary interaction vertex~\cite{pvtxreco} in
the event starts from the track collection.  Tracks are
first filtered with cuts on the impact parameter, number
of hits, and the $\chi^2$ per degree of freedom.  The
filtered tracks are then clustered in $z$.  Clusters are
split when there is a gap over a thresold. The tracks in the
cluster are then fit with an Adaptive Vertex Fit, where
tracks in the vertex are assigned a weight between 0 and 1
based on their compatibility with the common vertex.

The resolution on the primary vertex is tightly coupled to the impact
parameter resolution of a track. The primary vertex resolution is then
a strong function of the number of tracks used in fitting the vertex,
and the \pt of those tracks. 
It is not mathematically possible to define a primary vertex resolution 
for a specific sample since each vertex is independent and 
has its own uncertainty. 
Therefore, the best we can do is to try to measure an average 
resolution of an event ensemble.

In this note, we introduce the data-driven method, referred as two-vertex 
method, to measure the vertex resolution dependence on the 
number of tracks used in the primary vertex. 
In this method, we measure the primary vertex position of each event 
with two independent sets of tracks. 
We obtain the algorithm resolution by comparing the 
relative position of the two vertices in $x$, $y$ and $z$. 
The distribution of the difference in the fitted vertex positions 
can then be used to extract the resolution and pulls. 

