
An accurate determination of the interaction vertex in p-p collisions is crucial
to separate interesting \emph{hard interactions} of partons from the 
huge background due to long distance diffractive interact ions of protons. 
 To have an estimate of the performance of the Adaptive filter in collision data 
is thus fundamental for the most of physics analyses to be performed at CMS. 

The CMS collaboration have developed several primary vertex fitting algorithms~\cite{pvtxreco}, 
among which the \emph{Adaptive Vertex Filter} is the most robust one and 
used in default offline primary vertex reconstruction. 
It starts from the \emph{generalTrack} collection. 
The prompt tracks are selected with cuts on the transverse impact parameter, 
number of hits, and the $\chi^2$ per degree of freedom. 
These selected tracks are then clustered in $z$. Clusters are
split when there is a gap over a thresold. The tracks in each
cluster are then fit with an Adaptive Vertex Fit, which is 
based on an iterative re-weighted least-squares fit based on Kalman filter. 
Each track in the vertex is assigned a weight between 0 and 1
based on their compatibility with the common vertex.

The event primary vertex resolution is given by the uncertainty 
reported by the primary vertex algorithm. 
It is tightly coupled to the impact parameter resolution of a track. 
It also strongly depends on the number of tracks used in fitting the vertex,
and the \pt of those tracks. 
Since each vertex is independent and has its own uncertainty, it is not 
mathematically possible to define a primary vertex resolution for a specific sample.
Therefore, the best we can do is to try to measure an average 
resolution of an event ensemble.

In this note, we introduce the data-driven method, referred as two-vertex 
method, to measure the vertex resolution dependence on the 
number of tracks used in the primary vertex. 
In this method, we measure the primary vertex position of each event 
with two independent sets of tracks. 
We obtain the algorithm resolution by comparing the 
relative position of the two vertices in $x$, $y$ and $z$. 
The distribution of the difference in the fitted vertex positions 
can then be used to extract the resolution and pulls. 

