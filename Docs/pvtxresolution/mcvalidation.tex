
We can validate the procedure of the two-vertex method by comparing the 
results using the two-vertex method to the results obtained with MC method.
In Monte Carlo, the resolution of the primary vertex positions can be easily 
evaluated by comparing the reconstructed positions with the simulated 
positions. This is also referred as MC method later. In the MC method, 
the residual is defined as $x_{\tt rec}-x_{\tt sim}$. 

\subsection{Resolution and Pulls vs Number of Tracks}

Figure~\ref{fig:resvsntrk-900MC}-~\ref{fig:pullvsntrk-900MC} show 
the resolution and pull as a function of the nubmer of tracks used 
in the primary vertex for four different methods. 
In each plot, the results of the following four methods are overlaid:
%%%%%%%%%%%%%%%%%%
\begin{itemize}
\item Black: MC method applied on the unsplit vertex collection
\item Red: MC method applied on the first split vertex collection
\item Green: MC method results applied on the second split vertex collection
\item Blue: Two-vertex method results
\end{itemize}
%%%%%%%%%%%%%%%%%%

We see good agreement between the resolution obtained using MC method and 
two-vertex method. There is slight difference at large number of track bins ($>$10) 
between the resolutions obtained by applying MC method to 
the unsplit trackset and the rest of the methods. 
It is because of the <\pt> of the split track sets are in general 
100\,MeV larger than the original {\it generalTracks}. 

%%%%%%%%%%%%%%%%%%%%%%%
\begin{figure}[htb]
\begin{center}
\centerline{
\epsfig{figure=figures/ResX_900GeVMC.eps,height=2.0in}
\epsfig{figure=figures/ResY_900GeVMC.eps,height=2.0in}
\epsfig{figure=figures/ResZ_900GeVMC.eps,height=2.0in}
}
\caption{\sl
Primary vertex resolution as a function of the number of tracks used in the fitted vertex. 
}
\label{fig:resvsntrk-900MC}
\end{center}
%\end{figure}
%%%%%%%%%%%%%%%%%%%%%%%
%%%%%%%%%%%%%%%%%%%%%%%
%\begin{figure}[htb]
\begin{center}
\centerline{
\epsfig{figure=figures/PullX_900GeVMC.eps,height=2.0in}
\epsfig{figure=figures/PullY_900GeVMC.eps,height=2.0in}
\epsfig{figure=figures/PullZ_900GeVMC.eps,height=2.0in}
}
\caption{\sl
Fitted pulls from primary vertex distribution as a function of the number of tracks used in the 
fitted vertex.}
\label{fig:pullvsntrk-900MC}
\end{center}
\end{figure}
%%%%%%%%%%%%%%%%%%%%%%%


\subsection{Resolution and Pulls vs Number of Tracks at Different \pt\, Ranges}

{\it Actiion Item for PISA.}