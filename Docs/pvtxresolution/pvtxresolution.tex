% latex pvtxresolution ; dvips -Ppdf -t letter -G0 pvtxresolution
% ps2pdf -sPAPERSIZE=a4 pvtxresolution.ps
% preparation guidelines:
% https://twiki.cern.ch/twiki//bin/view/CMS/Internal/PubPreparation

\documentclass{cmspaper}
\usepackage{graphicx}
\usepackage{rotate}
\usepackage{relsize}
\usepackage{epsfig}
\usepackage{lineno}
\usepackage{verbatim} 
\setlength{\topmargin}{0.7in}

\newcommand{\pt} {\ensuremath{P_T}}
\newcommand{\ptm} {\ensuremath{\bar{P}_T}}

\begin{document}

%==============================================================================
% title page for many authors
%
% select one of the following and type in the proper number:
% \cmsnote{2005/000}
% \internalnote{2009/version1a}
\begin{titlepage}
\cmsnote{2010/version1}
\date{\today}

\title{Primary Vertex Resolution Measured with Data Driven Two-Vertex Method at $\sqrt{s} = $ 0.9/2.36 TeV}

  \begin{Authlist}
    J.~ Bernardini, F.~ Fiori, F.~ Palla
    \Instfoot{pisa}{INFN Sezione di Pisa, Pisa, Italy}    
    K.~Burkett, Y.~Gao
    \Instfoot{fnal}{Fermi National Accelerator Laboratory, Batavia, USA}
    G.~ Hanson, G.Y.~ Jeng
    \Instfoot{ucriverside}{University of California, Riverside, Riverside, USA}

  \end{Authlist}

\begin{abstract}

This note introduces a data driven algorithm, referred as two-vertex method, to 
measure the primary vertex spatial resolution. 
The resolution is measured as a function of the number of tracks used in fitting the 
vertex and the average \pt of those tracks. 
%The method is based on the application of the \emph{Adaptive Vertex Filter}, the most robust vertex finding algorithm developed by the CMS collaboration, on two independent sets of tracks belonging to the same hard interaction.
The method is validated on Monte Carlo simulated data and applied to 
the first LHC collision data collected at $\sqrt{s}$ = 0.9 and 2.36 TeV in CMS. 
The primary vertex resolution in mininum bias data is measured to be 
close to 50 $\mu m$ in $x$ and $y$, and 70 $\mu m$ in $z$. 
\end{abstract}
\end{titlepage}

\setcounter{page}{2}%JPP
\tableofcontents
\pagebreak


%------------------
\section {Introduction}
\label{sec:intro}
For the cross section measurement of $pp\to WW\to 2l 2\nu_l$
~\cite{wwanalysis}, one of the dominant backgrounds is the top-background. 
The main difference between the top-background and $WW$ signal is the 
presence of one or two extra $b$-jets in the final states. Thus, one 
efficienct way to suppress the top-background is to apply 
jet-veto, vetoing the events with leading jet \pt\, above a thresold at a 
given $\eta$ range. The specific \pt\, thresolds and the $\eta$ 
ranges are discussed in this note. 
%In $WW$ analysis, we choose 25 GeV as the working point. 

The jet-veto efficiency of $WW$ can not be measured directly 
from the data. We have to use the MC to certain extent. 
%signal needs to be carefully evaluated 
%for $WW$ cross section measurement. 
In this note, we describe a partially data-driven approach to 
measure $WW$ jet-veto signal effcicieny in data. 
In this approach, we use the well-defined process 
$pp\to Z\to 2l$ as a control region to test the agreement 
in the jet spectrum between data and MC. 
We then estimate the jet-veto signal efficiency on $WW$ as the 
jet-veto efficiency of $pp\to Z\to 2l$ on data multiplied 
by the $WW/Z$ jet-veto efficiency ratio estimated on the Monte Carlo, 
shown in Equation~\ref{eq:wweff}. 
Table~\ref{tab:datasets} summarizes the data and MC samples used in this note. 

The uncertainties of jet-veto signal efficiency come from 
both the matrix element calculations % uncertainties and 
and the jet reconstruction. %uncertainties. 
In the leading order matrix element calculation, 
the $pp\to WW\to 2l2\nu$ process does not contain jet.
Extra jets are primarily due to the initial 
state radition. It makes the jet energy spectrum 
sensitive to the corrections beyond the LO. 
We could estimate this systematic error by 
comparing the predictions from various MC generators, 
such as, the Pythia, Madgraph and the MC@NLO. 
On the jet reconstruction side, the effects from the jet energy correction 
and its uncertainties can propagate towards the jet-veto efficiency 
estimation as well. Validation of the jet response between the 
data and MC is thus required.

The note is organized as follows. In section~\ref{sec:zeff}, we discuss the jet-veto 
efficiency measurement on $Z$ data, in which 
the choice of the jet-veto $\eta$ region is examined. 
In section~\ref{sec:wwzratio}, we describe the $WW/Z$ jet-veto 
efficiency ratio estimate and its uncertainties. 
In section~\ref{sec:jetresponse}, we discuss the effect of the 
jet energy response to the results. Finally, in section~\ref{sec:conclusion}, 
we present the jet-veto efficiency measurement for $WW$ in data. 


%%%%%%%%%%%%%%%%%%%%%%%
\begin{eqnarray}
\epsilon_{WW}^{data} 
%\equiv {\epsilon_Z^{data}} \times {\frac{\epsilon_{WW}^{data}}{\epsilon_Z^{data}}} 
\approx {\epsilon_Z^{data}} \times ({\epsilon_{WW}^{MC}}/{\epsilon_Z^{MC}}) = {\epsilon_Z^{data}} \times R_{WW/Z}^{MC}  
\label{eq:wweff}
\end{eqnarray}
%%%%%%%%%%%%%%%%%%%%%%%


%%%%%%%%%%%%%%%%%%%%%5
\begin{table}[htbp]
\caption{List of datasets} 
\begin{center}
\label{tab:datasets}
\begin{tabular}{c|c}
\hline
\hline
\multicolumn{2}{c}{Collision Data} \\ 
Luminosity & Dataset \\ %& nEvents & note \\
\hline
\vspace{-3mm}   &    \cr
\multirow{5}{*} {3.1} & /EG\_Run2010A-PromptReco-v4\_RECO/ \\
& /EG\_Run2010A-PromptReco-v4\_RECO/ \\
& /EG\_Run2010A-Jul16thReReco-v2\_RECO/ \\
& /EG\_Run2010A-Jun14thReReco\_v1\_RECO/ \\
& /MinimumBias\_Commissioning10-SD\_EG-Jun14thSkim\_v1\_RECO/ \\
\hline
\multicolumn{2}{c}{Monte Carlo} \\
\hline
\multirow{3}{*} {Pythia} & /Zee\_Spring10-START3X\_V26\_S09-v1/\\ 
& /Zmumu\_Spring10-START3X\_V26\_S09-v1/\\ 
& /WW\_Spring10-START3X\_V26\_S09-v1 \\ \hline
\multirow{2}{*} {Madgraph} & /ZJets-madgraph\_Spring10-START3X\_V26\_S09-v1 \\
& /VVJets-madgraph\_Spring10-START3X\_V26\_S09-v1/ \\ \hline
\multirow{2}{*} {MC@NLO} & /Zgamma\_ee\_M20-mcatnlo\_Spring10-START3X\_V26\_S09-v1/ \\
& /Zgamma\_mumu\_M20-mcatnlo\_Spring10-START3X\_V26\_S09-v1/ \\
& /WWtoEE-mcatnlo\_Spring10-START3X\_V26\_S09-v1/ \\
& /WWtoEPlusMuMinus-mcatnlo\_Spring10-START3X\_V26\_S09-v1/ \\
& /WWtoEPlusTauMinus-mcatnlo\_Spring10-START3X\_V26\_S09-v1/ \\
& /WWtoMuMu-mcatnlo\_Spring10-START3X\_V26\_S09-v1/ \\
& /WWtoMuPlusEMinus-mcatnlo\_Spring10-START3X\_V26\_S09-v1/ \\
& /WWtoMuPlusTauMinus-mcatnlo\_Spring10-START3X\_V26\_S09-v1/ \\
& /WWtoTauTau-mcatnlo\_Spring10-START3X\_V26\_S09-v1/ \\
& /WWtoTauPlusEMinus-mcatnlo\_Spring10-START3X\_V26\_S09-v1/ \\
& /WWtoTauPlusMuMinus-mcatnlo\_Spring10-START3X\_V26\_S09-v1/ \\
\hline
\hline
\end{tabular}
\end{center}
\end{table}
%%%%%%%%%%%%%%%%%%%%%5


%------------------
\section{Data Samples and Event Selections}
\label{sec:datasample}
This analysis is performed with the data collected at the center-of-mass energy of 900\,GeV and 2360\,GeV. 

\subsection {Data Samples}

We use {\em December $19^{\rm th}$} reprocessing of the collected data and MC datasets:
\begin{itemize}
\item /MinimumBias/BeamCommissioning09-BSCNOBEAMHALO-Dec19thSkim\_336p3\_v1/
\item /MinBias/Summer09-STARTUP3X\_V8K\_900GeV-v1/
\item /MinBias/Summer09-STARTUP3X\_V8L\_2360GeV-v1/
\end{itemize} 

For the collision dataset, the alignment parameters of the silicon tracker were computed with about 
two million of cosmic ray tracks collected in November 2009 and the nominal values of the 
alignment parameter errors (APE) have been used in the reconstruction.
The simulated events used in this note are minimum-bias events produced with PYTHIA 6.4 \cite{Pythia} 
event generator at center-of-mass energies of 900\,GeV and 2360\,GeV and processed 
with a simulation of the CMS detector response based on GEANT 4 \cite{Geant}. 
The applied misalignment, miscalibration 
and dead channel map correspond to the expected start-up conditions. 
The longitudinal distribution of the primary collision vertices has been tuned to match the real
data. The signal in the silicon strip tracker was simulated in {\em peak} mode 
in agreement with the mode used in the readout chips, during the data taking~\cite{CMS_NOTE_2009_021}.  

To select the minimum bias collsion events, the skimmed collision dataset has the following 
technical trigger bits requirements: 
%the presence of at 
%least one BSC hit on both side of the CMS detector along the beam line 
%in coincidence with signals from both the BPTX detectors which indicate 
%the passage of a bunch from both beams at the interaction point. 
%The background from non-collision
%events has been reduced by requiring the absence of any coincidence in the BSC
% compatible with beam-produced particles which cross the detector
%from one end to the other. The trigger requirement is as follows:
\begin{itemize}
\item BSC trigger: technical trigger bit 40 or 41 
\item Veto BeamHalo: Triggers: 36, 37, 38, 39
\end{itemize}

In addition, we apply technical trigger bit 0 to pick up the correct bunch crossing for data. 
Note that this is not simulated in MC. 

\subsection{Event Selection}

To reduce further the background from non-collision events and to select useful events for tracking studies, 
we have the following event selections: 
\begin{itemize}
\item At least one real primary vertex reconstructed
\item fraction of {\em highPurity} tracks larger than 20\% if the 
number of reconstructed tracks is larger than 10.
\end{itemize}

Table~\ref{tab:datasets} shows the selection effcidency and final number of events analysis 
in each datasets.

%%%%%%%%%%%%%%%%%%%%%5
\begin{table}[htbH]
\begin{center}
\caption{List of datasets and selection efficiency. 
The selection efficiency on n-th colomn is obtained in addition to the cuts in the preceding n-1 columns.
\label{tab:datasets}}
\begin{tabular}{ccccc}
\hline
\hline
 Dataset name & Trigger & primary vertex & track high purity &  Events left \\
\hline
900\,GeV Dec19thSkim& 100\% & 94\% & 100\% & 260,000 \\
900\,GeV STARTUP3X\_V8K & 67\% & 93\% & 100\% & 256,000 \\
2360\,GeV Dec19thSkim & 100 \% & 95\% & 100\% & 12,959 \\
2360\,GeV STARTUP3X\_V8L & 68\% & 94\% & 100\% & 12,491 \\
\hline
\hline
\end{tabular}
\end{center}
\end{table}


%------------------
\section{Primary Vertex Resolution Measured with Two-Vertex Method}
\label{sec:twovertex}
\subsection{Two-vertex Method Description}

To measure the primary vertex resolution, we measure the primary vertex 
position of each event with two independent sets of tracks. 
We obtain the algorithm resolution by comparing the 
relative position of the two vertices in $x$, $y$ and $z$. 
The distribution of the difference in the fitted vertex positions 
can then be used to extract the resolution and pulls. 
The method proceeds as follows. 

%%%%%%%%%%%%%%%%%%%%
\begin{enumerate}
\item Divide tracks used in the primary vertex in two independent sets (A and B)
%%%%%%%%%%%%%%%%
\begin{itemize}
\item
We take tracks used the ``offlinePrimaryVertices'' and sort them in descending order of \pt. 
\item 
Starting from highest-\pt track, we take pairs of tracks and 
randomly choose which track goes into which set (A or B). This 
ensures that there is no asymmetry of the track \pt\, spectrum in 
the two vertices. If the number of the tracks are odd, we drop 
the lowest-\pt track. 
\end{itemize}
%%%%%%%%%%%%%%%%
We monitor the basic track distributions (\pt, $\eta$, $\phi$, $dxy$ and $dz$) 
of the split track sets. Figure~\ref{fig:splittrack} show the comparison 
of these variables between the original tracks and split track sets A and B in 
the mininum bias collision data taken at $\sqrt{s}=$900GeV. 
We see that the split track sets have consistent kinematics as the 
original tracks. 

%%%%%%
\item Vertex fit each trackset (A and B) independently using ``offlinePrimaryVertices'' algorithm.

\item Compare the primary vertex position between the two fit vertices and extract resolution and pull. 

When comparing the two fitted vertices, it is important that 
the number of tracks used in the final fitted vertices and the 
$\Sigma\pt$ and $z$ of the two vertices do not differ by too much. 
As we sort the tracks according the \pt\, prior to the splitting 
step, the difference of the $\Sigma\pt$ between the two vertices 
are negligible. We select on the following two variables to ensure the 
number of tracks and vertex $z$ difference of the two vertex is small:
\begin{itemize}
\item Relative difference of the number of tracks used in the two vertices, 
$|\frac{\tt{nTrk(vtx1)-nTrk(vtx2)}}{\tt{nTrk(vtx1)+nTrk(vtx2)}}|<0.1$
\item The separation signficance in z, defined as 
$\frac{|z(vtx1)-z(vtx2)|}{\tt{max}(\sigma z(vtx1),\sigma z(vtx2))}>5$
\end{itemize}
Figure~\ref{fig:anaselection} shows the distribution of these 
two variables before applying the cuts. The selection efficiency of the two 
cuts is found to be 74\% in both data and MC. 

We analyze the difference between the two selected matching vertices and 
and extract the resolution and pull dependence of the number of tracks 
used in the primary vertex as follows. 
%%%%%%%%%
\begin{itemize}
\item
The difference between the coordinates of the two vertices in $x$, $y$ 
and $z$ is histogrammed for each number of tracks. We fit a single 
gaussian to each distribution in the range of $\pm2\times$RMS. 
The resolution is defined as the gaussian width $\sigma$ divided by $\sqrt{2}$.
\item
The pull distribution at each number of track bin is filled with the 
quantity ${x_1-x_2}\over\sqrt{\sigma x_1^2+\sigma x_2^2}$, where 
$x_i$ and $\sigma x_i$ are the vertex position and errors of the two vertices.
We fit a single gaussian to each distribution in the range of $\pm2\times$RMS.
\end{itemize}
%%%%%%%%%
\end{enumerate}

Before moving forward, it is important to point out that this method rovides an estimate 
of the resolution and pulls comparing vertices formed 
with average number of tracks that are about half of the number of tracks 
that are used in the actual primary vertex fitter. 
As the intrinsic vertex resolution is well known to decrease with the 
number tracks used in the primary vertex fitter. Therefore, the average 
resolutin obtained from this method will always be larger than the actual 
resolution in a given sample. More details on the systematic uncertainties 
are described in section~\ref{sec:systematics}.

%%%%%%%%%%%%%%%%%%%%%%%
\begin{figure}[htb]
\begin{center}
\begin{tabular}{ccc}
\epsfig{figure=figures/trkPtPV.eps,height=2.0in}
\epsfig{figure=figures/trkEtaPV.eps,height=2.0in}
\epsfig{figure=figures/trkPhiPV.eps,height=2.0in}\\
\epsfig{figure=figures/trkDxyCorrPV.eps,height=2.0in}
\epsfig{figure=figures/trkDzPV.eps,height=2.0in}
\end{tabular}
\caption{\sl 
Track \pt, $\eta$, $\Phi$, $dxy$ and $dz$ distributions of the 
tracks used in the primary vertex. 
Data is shown with red dots while MC is shown as histogram filled in blue.
}
\label{fig:splittrack}
\end{center}
\end{figure}
%%%%%%%%%%%%%%%%%%%%%%%
%%%%%%%%%%%%%%%%%%%%%%%
\begin{figure}[htb]
\begin{center}
\centerline{
\epsfig{figure=figures/nTrkDiff.eps,height=2.5in}
\epsfig{figure=figures/twovtxzsign.eps,height=2.5in}
}
\caption{\sl
Left: The relative difference of number of tracks used in 
the two vertex fit; Right: The $z$-signficance of the 
two vertex fit. Please see text for the definition. 
Data is shown with red dots while MC is shown as histogram filled in blue.
}
\label{fig:anaselection}
\end{center}
\end{figure}
%%%%%%%%%%%%%%%%%%%%%%%

\subsection{Resolution and Pulls in Different \pt\, Ranges}


%------------------
\clearpage

%------------------
\section{Two-vertex Method Validation on MC Samples}
\label{sec:mcvalidation}

We can validate the procedure of the two-vertex method in MC data 
by comparing two-vertex method and MC method. 
In MC method, the primary vertex resolution is evaluated by 
comparing the reconstructed positions with the simulated 
positions $x_{\tt rec}-x_{\tt sim}$. 

\subsection{Resolution and Pulls vs Number of Tracks}

Figure~\ref{fig:resvsntrk-900MC}-~\ref{fig:pullvsntrk-900MC} show 
the resolution and pull as a function of the nubmer of tracks used 
in the primary vertex for four different methods. 
In each plot, the results of the following four methods are overlaid:
%%%%%%%%%%%%%%%%%%
\begin{itemize}
\item Black: MC method applied on the unsplit vertex collection
\item Red: MC method applied on the first split vertex collection
\item Green: MC method results applied on the second split vertex collection
\item Blue: Two-vertex method results
\end{itemize}
%%%%%%%%%%%%%%%%%%

We see good agreement between the resolution obtained using MC method and 
two-vertex method. There is slight difference at large number of track bins ($>$10) 
between the resolutions obtained by applying MC method to 
the unsplit trackset and the rest of the methods. 
It is because of the <\pt> of the split track sets are in general 
100\,MeV larger than the original {\it generalTracks}. 

%%%%%%%%%%%%%%%%%%%%%%%
\begin{figure}[htb]
\begin{center}
\centerline{
\epsfig{figure=figures/ResX_900GeVMC.eps,height=2.0in}
\epsfig{figure=figures/ResY_900GeVMC.eps,height=2.0in}
\epsfig{figure=figures/ResZ_900GeVMC.eps,height=2.0in}
}
\caption{\sl
Primary vertex resolution as a function of the number of tracks used in the fitted vertex. 
}
\label{fig:resvsntrk-900MC}
\end{center}
%\end{figure}
%%%%%%%%%%%%%%%%%%%%%%%
%%%%%%%%%%%%%%%%%%%%%%%
%\begin{figure}[htb]
\begin{center}
\centerline{
\epsfig{figure=figures/PullX_900GeVMC.eps,height=2.0in}
\epsfig{figure=figures/PullY_900GeVMC.eps,height=2.0in}
\epsfig{figure=figures/PullZ_900GeVMC.eps,height=2.0in}
}
\caption{\sl
Fitted pulls from primary vertex distribution as a function of the number of tracks used in the 
fitted vertex.}
\label{fig:pullvsntrk-900MC}
\end{center}
\end{figure}
%%%%%%%%%%%%%%%%%%%%%%%


\subsection{Resolution and Pulls vs Number of Tracks at Different \pt\, Ranges}

{\it Actiion Item for PISA.}
%------------------
\clearpage

%------------------
\section{Resolution and Pull on Collision Data}
\label{sec:dataresults}
\subsection{Resolution and Pull vs Number of Tracks}
\label{sub:900respull}

Figure~\ref{fig:resvsntrk-900} shows the measured primary vertex 
resolution as a function of the number of tracks in x (left), 
y (middle), and z (right). 
Results are shown for both the December data and the MC and 
a good agreement in the curves is seen. 
As shown in Figure~\ref{fig:errvsntrk-900}, we 
compare the two-vertex resolution with the default vertex 
error of``offlinePrimaryVertices'' using all tracks on the 
collision data. 
The two results match very well in the plateu region, with 
average difference less than 5 $\mu m$. 
Figure~\ref{fig:pullvsntrk-900} shows the measured pulls on the 
primary vertex, using the difference in the measured position and 
uncerntainties reported by the fit. 
The pulls are roughly flat and close to unity.

The primary vertex resolution measured are much smaller than the beam spot resontruction. 
As shown in Ref.~\cite{trk-10-001}, the transverse beam width is 
measured to be $\sigma_x \sim 200\mu m$ and $\sigma_y \sim 120\mu m$ 
in the 900GeV data and $\sigma_{x,y}\sim 120\mu m$ in the 
2.36 TeV data. The length in $z$ is found to be 
$\sigma_z\sim 4$ cm in the 900GeV data and $\sigma_z\sim 2.8$ cm in 
the 2.36TeV data.

%%%%%%%%%%%%%%%%%%%%%%%
\begin{figure}[htb]
\begin{center}
\centerline{
\epsfig{figure=figures/ResX_900GeV.eps,height=2.0in}
\epsfig{figure=figures/ResY_900GeV.eps,height=2.0in}
\epsfig{figure=figures/ResZ_900GeV.eps,height=2.0in}
}
\caption{\sl
Primary vertex resolution as a function of the number of tracks 
used in the fitted vertex.
}
\label{fig:resvsntrk-900}
\end{center}
%\end{figure}
%%%%%%%%%%%%%%%%%%%%%%%
%\begin{figure}[htb]
\begin{center}
\centerline{
\epsfig{figure=figures/VtxError_Data_X.eps,height=2.0in}
\epsfig{figure=figures/VtxError_Data_Y.eps,height=2.0in}
\epsfig{figure=figures/VtxError_Data_Z.eps,height=2.0in}
}
\caption{\sl
Comparison of the two-vertex resolution with the default vertex 
error of``offlinePrimaryVertices'' using all tracks on the 
collision data with $\sqrt{s}=900$GeV.
}
\label{fig:errvsntrk-900}
\end{center}
\end{figure}
%%%%%%%%%%%%%%%%%%%%%%%
%%%%%%%%%%%%%%%%%%%%%%%
\begin{figure}[htb]
\begin{center}
\centerline{
\epsfig{figure=figures/PullX_900GeV.eps,height=2.0in}
\epsfig{figure=figures/PullY_900GeV.eps,height=2.0in}
\epsfig{figure=figures/PullZ_900GeV.eps,height=2.0in}
}
\caption{\sl
Fitted pulls from primary vertex distribution as a function of the 
number of tracks used in the 
fitted vertex in 900GeV data. }
\label{fig:pullvsntrk-900}
\end{center}
\end{figure}
%%%%%%%%%%%%%%%%%%%%%%%



\subsection{Resolution and Pull vs Number of Tracks at Different \pt\, Ranges}
\label{sub:900respullpt}

Figures \ref{fig:respt} and \ref{fig:pullpt} show the resolution and pulls in three different \ptm ranges for MC and 900 GeV collisons data. The agreement with MC is very good for all the different \ptm bins. The pulls are falt and close to unity in the whole range.

%%%%%%%%%%%%%%%%%%%%%%%
\begin{figure}[!h]
\begin{center}
\centerline{
\epsfig{figure=figures/ResXpt.eps,height=2.0in}
\epsfig{figure=figures/ResYpt.eps,height=2.0in}
\epsfig{figure=figures/ResZpt.eps,height=2.0in}
}
\caption{\sl
Primary vertex resolution as a function of the number of tracks used in the fitted vertex in three different \ptm bins.
}
\label{fig:respt}
\end{center}
%\end{figure}
%%%%%%%%%%%%%%%%%%%%%%%
%%%%%%%%%%%%%%%%%%%%%%%
%\begin{figure}[htb]
\begin{center}
\centerline{
\epsfig{figure=figures/PullXpt.eps,height=2.0in}
\epsfig{figure=figures/PullYpt.eps,height=2.0in}
\epsfig{figure=figures/PullZpt.eps,height=2.0in}
}
\caption{\sl
Fitted pulls from primary vertex distribution as a function of the number of tracks used in the 
fitted vertex for three different \ptm bins.}
\label{fig:pullpt}
\end{center}
\end{figure}
%%%%%%%%%%%%%%%%%%%%%%%


\subsection{Compare Resolution and Pulls of 900 and 2360 GeV Data }
\label{sub:2360respull}

We apply the two-vertex method on the 2.36 TeV data. Figure~\ref{fig:resvsntrk-2360}-\ref{fig:pullvsntrk-2360} 
show the resolution and pulls results comparing $\sqrt{s}=900$GeV and 2360GeV data. 
We observe that the resolutions taken at these two energies are consistent. 

%%%%%%%%%%%%%%%%%%%%%%%
\begin{figure}[htb]
\begin{center}
\centerline{
\epsfig{figure=figures/ResX_2360GeV.eps,height=2.0in}
\epsfig{figure=figures/ResY_2360GeV.eps,height=2.0in}
\epsfig{figure=figures/ResZ_2360GeV.eps,height=2.0in}
}
\caption{\sl
Primary vertex resolution as a function of the number of tracks used in the 
fitted vertex.
}
\label{fig:resvsntrk-2360}
\end{center}
%\end{figure}
%%%%%%%%%%%%%%%%%%%%%%%
%%%%%%%%%%%%%%%%%%%%%%%
%\begin{figure}[htb]
\begin{center}
\centerline{
\epsfig{figure=figures/PullX_2360GeV.eps,height=2.0in}
\epsfig{figure=figures/PullY_2360GeV.eps,height=2.0in}
\epsfig{figure=figures/PullZ_2360GeV.eps,height=2.0in}
}
\caption{\sl
Fitted pulls from primary vertex distribution as a function of the number of tracks used in the 
fitted vertex.}
\label{fig:pullvsntrk-2360}
\end{center}
\end{figure}
%%%%%%%%%%%%%%%%%%%%%%%

%------------------
\clearpage

%------------------
\section{Systematic Effects and Cross Checks}
\label{sec:systematics}
In this section, we discuss briefly the systematic uncertainties in 
the primary vertex resolution measurement with two-vertex method.

\subsection{Effect of the Alignment Scenario}
\label{sub:alignsys}

The primary vertex resolution depends on the track impact parameter 
resolution, which is affected by the tracker alignment. 
At the early stage of data taking, the alignment senario may 
not describe the actual value. 
The systematic uncertainties on the track impact parameters induced 
by the misalignment will be translated to the systematic uncertainties
of primary vertex resolution. 

To estimate the effect of misalignment on the primary vertex resolution, 
we repeat the study on the MC data simulated with perfectly 
aligned tracker and with the alignment position error (APE) set to 0. 
Figure~\ref{fig:resvsntrk-aligncheck}-~\ref{fig:pullvsntrk-aligncheck} 
show the resolution and pulls versus number of tracks results comparing 
900 GeV collision data, MC with misalignment scenario matching 
the data taking and the MC with ideal geometry and APE.

From this study, we estimate systematic uncertainty due to the mis-alignment 
is within 5 $\mu m$ when the number of tracks exceed 5. 
The difference in resolution can be as large as around 20-30 $\mu m$ when 
number of tracks are less than 5. 
This difference also shows up in the corresponding bins of 
pull distributions. Besides the expected larger uncertainties on these 
low number of tracks vertices, we also find that the single 
gaussian fit does not describe the residual well. 
It is sensitive to the fit range. 
Nevertheless, in all number of track bins, the mis-aligned systematic 
uncertainties are within 10\% of the resolution itself. 

%%%%%%%%%%%%%%%%%%%%%%%
\begin{figure}[htb]
\begin{center}
\centerline{
\epsfig{figure=figures/ResX_IdealMC.eps,height=2.0in}
\epsfig{figure=figures/ResY_IdealMC.eps,height=2.0in}
\epsfig{figure=figures/ResZ_IdealMC.eps,height=2.0in}
}
\caption{\sl
Primary vertex resolution as a function of the number of tracks used in the 
fitted vertex, comparing the collison data taken at $\sqrt{s}=900$GeV (Black), 
mininum bias MC with startup (green) and ideal (red) alignment. }
\label{fig:resvsntrk-aligncheck}
\end{center}
%\end{figure}
%%%%%%%%%%%%%%%%%%%%%%%
%%%%%%%%%%%%%%%%%%%%%%%
%\begin{figure}[htb]
\begin{center}
\centerline{
\epsfig{figure=figures/PullX_IdealMC.eps,height=2.0in}
\epsfig{figure=figures/PullY_IdealMC.eps,height=2.0in}
\epsfig{figure=figures/PullZ_IdealMC.eps,height=2.0in}
}
\caption{\sl
Fitted pulls from primary vertex distribution 
as a function of the number of tracks used 
in the fitted vertex,  comparing the collison data taken at $\sqrt{s}=900$GeV (Black), 
mininum bias MC with startup (green) and ideal (red) alignment. }
\label{fig:pullvsntrk-aligncheck}
\end{center}
\end{figure}
%%%%%%%%%%%%%%%%%%%%%%%


\subsection{Effect of the split vertex weight difference}
\label{sub:weight}

In the Adaptiver filter, all tracks in a given cluster are used in the 
vertex fit. The outlier tracks with larger distance to the vertex postion are 
down weighted significantly thus contribute little to the fitted vertex 
positions or errors. 
Thus as we compare the two split vertices, though the number of tracks do not 
differ, the number of tracks with high weights ($>0.5$) (or the numbers of degree of freedom) may differ. 
In that case, the difference between the two vertex positions does not 
represent the vertex error in either vertex. 

To cross check this effect, we repeat the study with resolution 
and pull versus the number of high weighted tracks instead of the total number 
of tracks. 
Figure~\ref{fig:resvsntrk-HW}-~\ref{fig:pullvsntrk-HW} show the comparison 
between the two approaches. 
The difference between the two methods are expected as the number of high 
weighted tracks is always smaller than the number of total tracks in the 
vertex. On the other hand, the resolution at the stable (or tail) region 
of number of tracks are consistent with each. This indicates that 
the low weight tracks do not affect the measurement. 
Besides, the pull distributions are consistent in all number of tracks bins.

%%%%%%%%%%%%%%%%%%%%%%%
\begin{figure}[htb]
\begin{center}
\centerline{
\epsfig{figure=figures/ResX_HW.eps,height=2.0in}
\epsfig{figure=figures/ResY_HW.eps,height=2.0in}
\epsfig{figure=figures/ResZ_HW.eps,height=2.0in}
}
\caption{\sl
Primary vertex resolution as a function of the number of tracks used in the 
fitted vertex.
}
\label{fig:resvsntrk-HW}
\end{center}
%\end{figure}
%%%%%%%%%%%%%%%%%%%%%%%
%%%%%%%%%%%%%%%%%%%%%%%
%\begin{figure}[htb]
\begin{center}
\centerline{
\epsfig{figure=figures/PullX_HW.eps,height=2.0in}
\epsfig{figure=figures/PullY_HW.eps,height=2.0in}
\epsfig{figure=figures/PullZ_HW.eps,height=2.0in}
}
\caption{\sl
Fitted pulls from primary vertex distribution as a function of the number of tracks used in the 
fitted vertex.}
\label{fig:pullvsntrk-HW}
\end{center}
\end{figure}
%%%%%%%%%%%%%%%%%%%%%%%
\clearpage
%------------------
\clearpage

\section{Conclusion}
\label{sec:conclusion}
In summary, we described a data-driven method to estimate the $WW$ 
jet-veto signal efficiency and its systematic uncertainties. 
The $WW$ jet-veto efficiency in data is parametrized as the product 
of the $Z$ jet-veto efficiency in the data and the $WW/Z$ jet-veto efficiency 
in the MC. 

We have measured the $Z$ jet-veto efficiencies in data and 
compared to various MC samples, shown in 
Table~\ref{tab:zeff_results}-\ref{tab:zeff_jec_results}. 
We observed good agreement of $Z$ jet-veto efficiency 
between data and the Pythia and Madgraph MC samples. 
%The jet-veto efficiency predicted by the MC@NLO MC is statistically 
%smaller than the values predicted by the Pythia and Madgraph MC. 
We have studied the $WW/Z$ jet-veto efficiency ratios using the 
Pythia, Madgraph and MC@NLO MC samples. 
The predictions from the Madgraph and MC@NLO MC samples are 
found to agree with each other within 1\%. 
The jet energy spectrum in the Pythia MC is however significantly softer 
than the other two, yielding a difference of more than 10\% at 
the jet \pt cut of 20 GeV. 
Based on the above findings, we choose the Madgraph as the 
nominal MC to calculate the $WW/Z$ jet-veto efficiency ratio 
and assign half of the difference between the Pythia and Madgraph 
predictions as the $R_{WW/Z}^{MC}$ uncertainties. 
%This large difference is likely due to the 
%account for the large angle initial state radiation 

The $WW$ jet-veto efficiencies using the uncorrected and 
corrected PFJet at \pt cut of (20, 25, 30) GeV are tabulated 
in Table~\ref{tab:wweff_results}-\ref{tab:wweff_jec_results}. 







%%%% Need a table for the final results

%%%%%%%%%%%%%%%%%%%%%5
\begin{table}[htbp]
\caption{Jet-veto signal efficiency on $WW$ using the uncorrected PFJet. Madgraph MC is chosen 
for the $R_{WW/Z}^{MC}$}
\begin{center}
\label{tab:wweff_results}
\begin{tabular}{c|cc|c}
\hline
\hline
Jet-veto Cut (GeV) & $\epsilon_Z^{data}$ (\%) & $R_{WW/Z}^{MC}$ (\%) & $\epsilon_{WW}$ (\%) \\
    20 &   76.3$\pm$ 1.0 &   76.3$\pm$ 0.7$\pm$ 6.2 &   58.3$\pm$ 0.9$\pm$ 4.8 \\
    25 &   82.6$\pm$ 0.9 &   79.1$\pm$ 0.6$\pm$ 4.9 &   65.4$\pm$ 0.9$\pm$ 4.1 \\
    30 &   87.2$\pm$ 0.8 &   80.9$\pm$ 0.6$\pm$ 4.2 &   70.5$\pm$ 0.8$\pm$ 3.7 \\
\hline
\hline
\end{tabular}
\end{center}
%\end{table}
%%%%%%%%%%%%%%%%%%%%%5
%%%%%%%%%%%%%%%%%%%%%5
%\begin{table}[htbp]
\caption{Jet-veto signal efficiency on $WW$ using the L2L3 corrected PFJet for both data and MC. 
For data, additional residual correction is applied as well. 
Madgraph MC is chosen for the $R_{WW/Z}^{MC}$}
\begin{center}
\label{tab:wweff_jec_results}
\begin{tabular}{c|cc|c}
\hline
\hline
Jet-veto Cut (GeV) & $\epsilon_Z^{data}$ (\%) & $R_{WW/Z}^{MC}$ (\%) & $\epsilon_{WW}$ (\%) \\
    20 &   71.3$\pm$ 1.1 &   74.9$\pm$ 0.7$\pm$ 6.5 &   53.4$\pm$ 1.0$\pm$ 4.6 \\
    25 &   78.8$\pm$ 1.0 &   77.5$\pm$ 0.7$\pm$ 5.7 &   61.1$\pm$ 0.9$\pm$ 4.5 \\
    30 &   84.5$\pm$ 0.9 &   79.7$\pm$ 0.6$\pm$ 4.8 &   67.3$\pm$ 0.9$\pm$ 4.1 \\
\hline
\hline
\end{tabular}
\end{center}
\end{table}
%%%%%%%%%%%%%%%%%%%%%5


% References
%-----------------------------------------------------------------------
\begin{thebibliography}{99}

\bibitem{wwanalysis}
CMS AN -2009/042, ``Prospects for measuring the $WW$ production 
cross section in $pp$ collisions at $\sqrt{s}=$ 10 TeV''

\bibitem{jetalgo}
CMS PAS JME-07-003, ``Performance of Jet Algorithms in CMS''

\bibitem{jet-7tev}
CMS PAS JME-10-003, ``Jet Performance in pp Collisions at $\sqrt{s}=7$ TeV''

\bibitem{jec-kostas}
https://hypernews.cern.ch/HyperNews/CMS/get/JetMET/1017.html

\bibitem{jec-zbalance}
CMS PAS JME-09-005, ``Determination of the jet energy scale using 
$Z\to e^+e^-$ + jet \pt balance and a procedure for combining data driven 
corrections'', 
CMS PAS JME-09-009, ``Calibration of the absolute jet energy scale with 
$Z(\to\mu^+\mu^-)$ + jet events at CMS''


\end{thebibliography}
%-----------------------------------------------------------------------


\end{document}


