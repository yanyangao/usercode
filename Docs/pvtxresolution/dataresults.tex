\subsection{Resolution and pull vs number of tracks}


Figure~\ref{fig:resvsntrk-900} shows the measured primary vertex 
resolution as a function of the number of tracks in x (left), 
y (middle), and z (right). 
Results are shown for both the December data and the MC and 
a good agreement in the curves is seen. 
Figure~\ref{fig:pullvsntrk-900} shows the measured pulls on the 
primary vertex, using the difference in the measured position and 
uncerntainties reported by the fit. 
The pulls are roughly flat and close to unity.

%%%%%%%%%%%%%%%%%%%%%%%
\begin{figure}[htb]
\begin{center}
\centerline{
\epsfig{figure=figures/ResX_900GeV.eps,height=2.0in}
\epsfig{figure=figures/ResY_900GeV.eps,height=2.0in}
\epsfig{figure=figures/ResZ_900GeV.eps,height=2.0in}
}
\caption{\sl
Primary vertex resolution as a function of the number of tracks used in the 
fitted vertex.
}
\label{fig:resvsntrk-900}
\end{center}
%\end{figure}
%%%%%%%%%%%%%%%%%%%%%%%
%%%%%%%%%%%%%%%%%%%%%%%
%\begin{figure}[htb]
\begin{center}
\centerline{
\epsfig{figure=figures/PullX_900GeV.eps,height=2.0in}
\epsfig{figure=figures/PullY_900GeV.eps,height=2.0in}
\epsfig{figure=figures/PullZ_900GeV.eps,height=2.0in}
}
\caption{\sl
Fitted pulls from primary vertex distribution as a function of the number of tracks used in the 
fitted vertex.}
\label{fig:pullvsntrk-900}
\end{center}
\end{figure}
%%%%%%%%%%%%%%%%%%%%%%%



\subsection{Resolution and pull vs number of tracks at different \pt\, ranges}

{\it Action Item for PISA}

\subsection{Compare resolution and pulls at $\sqrt{s}=$ 900 GeV and 2360GeV}

In this note, we describe the method and cross check with the mininum bias 900GeV results. 
We repeat the study on the data taken at $\sqrt{s}=2360$GeV as well. 
Figure~\ref{fig:resvsntrk-2360}-\ref{fig:pullvsntrk-2360} show the 
comparison results of resolution and pulls versus the number of tracks 
between the data taken at $\sqrt{s}=900$GeV and 2360GeV.

%%%%%%%%%%%%%%%%%%%%%%%
\begin{figure}[htb]
\begin{center}
\centerline{
\epsfig{figure=figures/ResX_2360GeV.eps,height=2.0in}
\epsfig{figure=figures/ResY_2360GeV.eps,height=2.0in}
\epsfig{figure=figures/ResZ_2360GeV.eps,height=2.0in}
}
\caption{\sl
Primary vertex resolution as a function of the number of tracks used in the 
fitted vertex.
}
\label{fig:resvsntrk-2360}
\end{center}
%\end{figure}
%%%%%%%%%%%%%%%%%%%%%%%
%%%%%%%%%%%%%%%%%%%%%%%
%\begin{figure}[htb]
\begin{center}
\centerline{
\epsfig{figure=figures/PullX_2360GeV.eps,height=2.0in}
\epsfig{figure=figures/PullY_2360GeV.eps,height=2.0in}
\epsfig{figure=figures/PullZ_2360GeV.eps,height=2.0in}
}
\caption{\sl
Fitted pulls from primary vertex distribution as a function of the number of tracks used in the 
fitted vertex.}
\label{fig:pullvsntrk-2360}
\end{center}
\end{figure}
%%%%%%%%%%%%%%%%%%%%%%%