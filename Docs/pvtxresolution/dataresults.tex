\subsection{Resolution and Pull vs Number of Tracks}
\label{sub:900respull}

Figure~\ref{fig:resvsntrk-900} shows the measured primary vertex 
resolution as a function of the number of tracks in x (left), 
y (middle), and z (right). 
Results are shown for both the December data and the MC and 
a good agreement in the curves is seen. 
As shown in Figure~\ref{fig:errvsntrk-900}, we 
compare the two-vertex resolution with the default vertex 
error of``offlinePrimaryVertices'' using all tracks on the 
collision data. 
The two results match very well in the plateu region, with 
average difference less than 5 $\mu m$. 
Figure~\ref{fig:pullvsntrk-900} shows the measured pulls on the 
primary vertex, using the difference in the measured position and 
uncerntainties reported by the fit. 
The pulls are roughly flat and close to unity.

The primary vertex resolution measured are much smaller than the beam spot resontruction. 
As shown in Ref.~\cite{trk-10-001}, the transverse beam width is 
measured to be $\sigma_x \sim 200\mu m$ and $\sigma_y \sim 120\mu m$ 
in the 900GeV data and $\sigma_{x,y}\sim 120\mu m$ in the 
2.36 TeV data. The length in $z$ is found to be 
$\sigma_z\sim 4$ cm in the 900GeV data and $\sigma_z\sim 2.8$ cm in 
the 2.36TeV data.

%%%%%%%%%%%%%%%%%%%%%%%
\begin{figure}[htb]
\begin{center}
\centerline{
\epsfig{figure=figures/ResX_900GeV.eps,height=2.0in}
\epsfig{figure=figures/ResY_900GeV.eps,height=2.0in}
\epsfig{figure=figures/ResZ_900GeV.eps,height=2.0in}
}
\caption{\sl
Primary vertex resolution as a function of the number of tracks 
used in the fitted vertex.
}
\label{fig:resvsntrk-900}
\end{center}
%\end{figure}
%%%%%%%%%%%%%%%%%%%%%%%
%\begin{figure}[htb]
\begin{center}
\centerline{
\epsfig{figure=figures/VtxError_Data_X.eps,height=2.0in}
\epsfig{figure=figures/VtxError_Data_Y.eps,height=2.0in}
\epsfig{figure=figures/VtxError_Data_Z.eps,height=2.0in}
}
\caption{\sl
Comparison of the two-vertex resolution with the default vertex 
error of``offlinePrimaryVertices'' using all tracks on the 
collision data with $\sqrt{s}=900$GeV.
}
\label{fig:errvsntrk-900}
\end{center}
\end{figure}
%%%%%%%%%%%%%%%%%%%%%%%
%%%%%%%%%%%%%%%%%%%%%%%
\begin{figure}[htb]
\begin{center}
\centerline{
\epsfig{figure=figures/PullX_900GeV.eps,height=2.0in}
\epsfig{figure=figures/PullY_900GeV.eps,height=2.0in}
\epsfig{figure=figures/PullZ_900GeV.eps,height=2.0in}
}
\caption{\sl
Fitted pulls from primary vertex distribution as a function of the 
number of tracks used in the 
fitted vertex in 900GeV data. }
\label{fig:pullvsntrk-900}
\end{center}
\end{figure}
%%%%%%%%%%%%%%%%%%%%%%%



\subsection{Resolution and Pull vs Number of Tracks at Different \pt\, Ranges}
\label{sub:900respullpt}

Figures \ref{fig:respt} and \ref{fig:pullpt} show the resolution and pulls in three different \ptm ranges for MC and 900 GeV collisons data. The agreement with MC is very good for all the different \ptm bins. The pulls are falt and close to unity in the whole range.

%%%%%%%%%%%%%%%%%%%%%%%
\begin{figure}[!h]
\begin{center}
\centerline{
\epsfig{figure=figures/ResXpt.eps,height=2.0in}
\epsfig{figure=figures/ResYpt.eps,height=2.0in}
\epsfig{figure=figures/ResZpt.eps,height=2.0in}
}
\caption{\sl
Primary vertex resolution as a function of the number of tracks used in the fitted vertex in three different \ptm bins.
}
\label{fig:respt}
\end{center}
%\end{figure}
%%%%%%%%%%%%%%%%%%%%%%%
%%%%%%%%%%%%%%%%%%%%%%%
%\begin{figure}[htb]
\begin{center}
\centerline{
\epsfig{figure=figures/PullXpt.eps,height=2.0in}
\epsfig{figure=figures/PullYpt.eps,height=2.0in}
\epsfig{figure=figures/PullZpt.eps,height=2.0in}
}
\caption{\sl
Fitted pulls from primary vertex distribution as a function of the number of tracks used in the 
fitted vertex for three different \ptm bins.}
\label{fig:pullpt}
\end{center}
\end{figure}
%%%%%%%%%%%%%%%%%%%%%%%


\subsection{Compare Resolution and Pulls of 900 and 2360 GeV Data }
\label{sub:2360respull}

We apply the two-vertex method on the 2.36 TeV data. Figure~\ref{fig:resvsntrk-2360}-\ref{fig:pullvsntrk-2360} 
show the resolution and pulls results comparing $\sqrt{s}=900$GeV and 2360GeV data. 
We observe that the resolutions taken at these two energies are consistent. 

%%%%%%%%%%%%%%%%%%%%%%%
\begin{figure}[htb]
\begin{center}
\centerline{
\epsfig{figure=figures/ResX_2360GeV.eps,height=2.0in}
\epsfig{figure=figures/ResY_2360GeV.eps,height=2.0in}
\epsfig{figure=figures/ResZ_2360GeV.eps,height=2.0in}
}
\caption{\sl
Primary vertex resolution as a function of the number of tracks used in the 
fitted vertex.
}
\label{fig:resvsntrk-2360}
\end{center}
%\end{figure}
%%%%%%%%%%%%%%%%%%%%%%%
%%%%%%%%%%%%%%%%%%%%%%%
%\begin{figure}[htb]
\begin{center}
\centerline{
\epsfig{figure=figures/PullX_2360GeV.eps,height=2.0in}
\epsfig{figure=figures/PullY_2360GeV.eps,height=2.0in}
\epsfig{figure=figures/PullZ_2360GeV.eps,height=2.0in}
}
\caption{\sl
Fitted pulls from primary vertex distribution as a function of the number of tracks used in the 
fitted vertex.}
\label{fig:pullvsntrk-2360}
\end{center}
\end{figure}
%%%%%%%%%%%%%%%%%%%%%%%
